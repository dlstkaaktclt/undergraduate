\documentclass{article}
\usepackage{bm}
\usepackage{amsmath}
\usepackage{graphicx}
\usepackage{mdwlist}
\usepackage[colorlinks=true]{hyperref}
\usepackage{geometry}
\geometry{margin=1in}
\geometry{headheight=2in}
\geometry{top=2in}
\usepackage{palatino}
%\renewcommand{\rmdefault}{palatino}
\usepackage{fancyhdr}
\usepackage{kotex}
%\pagestyle{fancy}
\rhead{}
\lhead{}
\chead{%
  {\vbox{%
      \vspace{2mm}
      \large
      실험 계획 및 실습
      \\[4mm]
      Homework \#(\textbf{2})\\
      \textbf{2014-16757 김보창}
    }
  }
}
\graphicspath{
    {C:/image/}
}

\usepackage[svgnames]{xcolor}
\usepackage{listings}

\definecolor{codegreen}{rgb}{0,0.6,0}
\definecolor{codegray}{rgb}{0.5,0.5,0.5}
\definecolor{codepurple}{rgb}{0.58,0,0.82}
\definecolor{backcolour}{rgb}{0.95,0.95,0.92}

\lstdefinestyle{mystyle}{
    backgroundcolor=\color{backcolour},   
    commentstyle=\color{codegreen},
    keywordstyle=\color{magenta},
    numberstyle=\tiny\color{codegray},
    stringstyle=\color{codepurple},
    basicstyle=\ttfamily\footnotesize,
    breakatwhitespace=false,         
    breaklines=true,                 
    captionpos=b,                    
    keepspaces=true,                 
    numbers=left,                    
    numbersep=5pt,                  
    showspaces=false,                
    showstringspaces=false,
    showtabs=false,                  
    tabsize=2
}

\lstset{style=mystyle}

\lstset{ 
  language=R,                     % the language of the code
  basicstyle=\ttfamily, % the size of the fonts that are used for the code
  numbers=left,                   % where to put the line-numbers
  numberstyle=\tiny\color{blue},  % the style that is used for the line-numbers
  stepnumber=1,                   % the step between two line-numbers. If it is 1, each line
                                  % will be numbered
  numbersep=5pt,                  % how far the line-numbers are from the code
  backgroundcolor=\color{white},  % choose the background color. You must add \usepackage{color}
  showspaces=false,               % show spaces adding particular underscores
  showstringspaces=false,         % underline spaces within strings
  showtabs=false,                 % show tabs within strings adding particular underscores
  frame=single,                   % adds a frame around the code
  rulecolor=\color{black},        % if not set, the frame-color may be changed on line-breaks within not-black text (e.g. commens (green here))
  tabsize=2,                      % sets default tabsize to 2 spaces
  captionpos=b,                   % sets the caption-position to bottom
  breaklines=true,                % sets automatic line breaking
  breakatwhitespace=false,        % sets if automatic breaks should only happen at whitespace
  keywordstyle=\color{RoyalBlue},      % keyword style
  commentstyle=\color{YellowGreen},   % comment style
  stringstyle=\color{ForestGreen}      % string literal style
} 



\usepackage{paralist}
\usepackage{todonotes}
\setlength{\marginparwidth}{2.15cm}

\usepackage{tikz}
\usetikzlibrary{positioning,shapes,backgrounds}

\begin{document}
\pagestyle{fancy}


%% Q1
\section{Q1} 

3.26

여기서 모형은 
$y_{ij} = \mu + \tau_i + \epsilon_{ij} = \mu_i + \epsilon_{ij} $, $i = 1...3, j = 1, 2... 5$, $\sum_{i=1}^3 \tau_i = 0 $으로 놓을 수 있다.

($\epsilon_{ij}  \overset{\text{i.i.d.}}{\sim} N(0, \sigma^2)$)


\subsection{1-(a)}

$$H_0 : \tau_i = 0 , \forall i = 1, 2, 3$$ 
$$H_1 :  \tau_i \neq 0, \exists i$$

로 귀무가설과 대립가설을 세우고, single factor ANOVA test를 진행하면 된다.

우리는 $H_0$하에서 $\frac{MS_{Trt}}{MS_E} \sim F_{a-1, N-a} $임을 알고 있으므로,

$F_0 = \frac{MS_{Trt}}{MS_E}$ 로 두고,  유의수준 $\alpha$에서 $F_0 > F_{a-1, N-a}(\alpha)$이면 귀무가설을 기각할 것이다.

이를 구하기 위해 $F_0$의 값을 구할것인데, 계산을 쉽게 하기 위해 R을 이용할 것이다.

다음 R코드를 이용하여 $F_0$의 값을 구한다.


\begin{lstlisting}[language=R]
y1 <- c(100, 96, 92, 96, 92)
y2 <- c(76, 80, 75, 84, 82)
y3 <- c(108, 100, 96, 98, 100)
y <- c(y1, y2, y3)

group <- rep(1:3, c(5,5,5))
group_df <- data.frame(y, group)
group_df <- transform(group_df, group = as.factor(group))

result <- summary(aov(y ~ group, data = group_df))
\end{lstlisting}

데이터를 입력해주고, group이라는 벡터에 각 데이터가 어떤 그룹의 데이터인지 명시해준다.

그후, data.frame 함수를 이용해 data frame으로 만들어주고, transform함수를 통해 group을 factor형으로 바꿔준다.

이제, aov와 summary함수를 이용하여 결과를 출력하면 다음과 같은 값이 나온다.

\begin{center}
    \includegraphics{hw2_1_aov.jpg}
\end{center} 

이를통해, 여기서의 $MS_E$의 값은 15.6, $MS_{Trt}$의 값은 598.1임을 알 수 있고,

$F_0 = 38.34$임을 알 수 있다. 또한 이때의 P-value가  $6.14 * 10^{-6}$임을 알 수 있다.

따라서, $\alpha = 0.05$로 택했을때, P-value가 0.05보다 작으므로 귀무가설을 기각할 수 있다.

즉, 평균 배터리 수명이 다르다는 결론을 내리게 된다.
\\

\subsection{1-(b)}

residual에 대한 분석을 하기 위해, 실습시간에 배운 R함수를 이용한다.

\begin{lstlisting}[language=R]
opar <- par(mfrow=c(2,2),cex=.8)
plot(aov(y ~ group, data = group_df))
\end{lstlisting}

첫줄을 통해 그래프를 출력할 환경을 지정해주고,
두번째줄 plot을 이용하여 그래프를 출력하게 하였다.

결과는 다음과 같다.

\begin{center}
    \includegraphics{hw2_1_resi.jpg}
\end{center} 

왼쪽 위 그래프를 보면, fitted value와 residual로 그래프를 그렸을때, 특정 경향성이 나타나지 않는것을 알 수 있다. 

따라서 우리 모형이 데이터를 잘 표현한다고 말할 수 있고,

오른쪽 아래 그래프를 봐도, factor level에 따라 residual의 차이가 거의 없으므로

등분산 가정을 위반하지 않음을 알 수 있다.

normal qq 그래프를 그려도, residual의 경향이 직선으로 정렬된 것을 보아 normality 가정을 위반하지 않음을 알 수 있다.

따라서 $\epsilon_{ij} \sim N(0,\sigma^2)$이라는 우리의 가정에 문제가 없음을 알 수 있다.

\subsection{1-(c)}

먼저 배터리 2의 confidence interval을 구하자.

그전에, ANOVA모형으로 잠시 되돌아가서, $MS_E = \frac{SS_E}{N-a}$임을 알고 있고,

$$SS_E = \sum_{i=1}^a \sum_{j=1}^n (y_{ij} - \bar{y_{i.}})^2$$,

$$\frac{SS_E}{\sigma^2} \sim \chi_{N-a}$$ 임을 알고있다.

또한, 우리는 $\bar{y}$가 표본평균, $S^2$가 표본분산이라 할때,  $\bar{y}$와 $S^2$가 independent함을 알고 있다.

때문에, $\bar{y_{i.}}$, $S_i^2$는 서로 indep하고, $SS_E = (n-1)\sum_{i=1}^a S_i^2$ 임을 안다.

이때, $\epsilon_{ij}$가 서로 indep하므로, $i \neq j$일때 $\bar{y_{i.}}$와 $S_j^2$가 indep함은 자명하다.

($\bar{y_{i.}}$를 이루는 항들은 $S_j^2$를 계산하는데 전혀 사용되지 않음)

따라서, 모든 (j = 1, 2, ... a)에 대해 $\bar{y_{i.}}$와 $S_j^2$ 가 indep 하므로, $\bar{y_{i.}}$ 와 $SS_E$는 indep하다.


그러므로 이제 $u_i$의 confidence interval을 구할 수 있다.

$\bar{y_{i.}} \sim N(\mu_i, \frac{\sigma^2}{n})$ 에서

$$\frac{\bar{y_{i.}} - \mu_i}{\frac{\sigma}{\sqrt{n}}} \sim N(0,1)$$,

$$SS_E/\sigma^2 \sim \chi_{N-a}$$

$$T = \frac{\bar{y_{i.}} - \mu_i}{\sqrt{\frac{MS_E}{n}}} \sim t_{N-a}$$가 되므로,

따라서, 이를 이용해서 $\mu_i$의 $100(1-\alpha)\%$ confidence interval을 구하면

$P(|T| \le t_{N-a}(\frac{\alpha}{2})) = 1-\alpha$에서,

$$\bar{y_{i.}} \pm t_{N-a}(\frac{\alpha}{2}) \sqrt{\frac{MS_E}{n}}$$ 가 된다.

마찬가지로, $\bar{y_{i.}} - \bar{y_{j.}}$ 은 $\bar{y_{i.}}$, $\bar{y_{j.}}$가 각각 $SS_E$와 독립이므로,

$SS_E$와 독립이 되고,

$$\bar{y_{i.}} - \bar{y_{j.}} ~ N(\mu_i - \mu_j, \frac{2\sigma^2}){n}$$ 에서,

같은 방법으로 $\mu_i - \mu_j$의 $100(1-\alpha)\%$ confidence interval을 구하면

$$\bar{y_{i.}} - \bar{y_{j.}} \pm t_{N-a}(\frac{\alpha}{2}) \sqrt{\frac{2 MS_E}{n}}$$


따라서, 위 계산과정을 이용하여 R 코드를 이용하여 답을 구하면,

\begin{lstlisting}[language=R]
MSE <- result[[1]][2,3]
n <- 5
mean2 <- mean(y2)
mean3 <- mean(y3)
t1 <- c(-1, 1) * qt(0.975, 12)
t2 <- c(-1, 1) * qt(0.995, 12)
int1 <- mean2 + t1 * sqrt(MSE/n)
int2 <- mean2 - mean3 + t2 * sqrt(2*MSE/n)
\end{lstlisting}

\begin{center}
    \includegraphics{hw2_1_ci.jpg}
실행 결과
\end{center} 

따라서 배터리 브랜드 2 평균 수명의 $95 \%$ CI는 다음과 같다.
$$[75.55145, 83.24855]$$

마찬가지로, 브랜드 2 평균수명 - 브랜드 3 평균수명의 $99 \%$ CI는 다믕과 같다.
$$[-28.63024, -13.36976]$$


\subsection{1-(d)}
각 브랜드의 평균은 다음과 같다.

\begin{center}
    \includegraphics{hw2_1_mean.jpg}
\end{center} 

보다시피, 3번브랜드의 배터리 수명 표본 평균이 100.4로 가장 높으므로,3번 브랜드를 택할것이다.

이제, 3번브랜드의 배터리 수명 평균, 즉 ,$\mu_3 = 100.4$로 가정하고, 1-(a)에서 $MS_E = 15.6$임을 알 수 있으므로,

$\sigma = \sqrt{MS_E} = \sqrt{15.6}$로 가정한다.

그리고 3번 브랜드의 배터리 수명이 normal분포를 따른다고 가정하면, 이제 배터리의 수명이 85보다 작을 확률을 구할 수 있다.

$y \sim N(100.4 , 15.6)$에서, $P(y < 85)$를 구하면 

$P(y < 85) = P(Z < \frac{(85 - 100.4)}{\sqrt{15.6}})$의 값을 구하면 되고, 이는 아래와 같이 구해진다.

\begin{center}
    \includegraphics{hw2_1_norm.jpg}
\end{center} 
즉, $4.83 * 10^{-5}$ 정도의 확률을 가짐을 알 수 있다.


  





\section{Q2} 

3.22에 있는 데이터는 circuit type에 따른 response time을 나타낸것으로, 다음과 같은 ANOVA 모델을 세울 수 있다.

$$y_{ij} = \mu + \tau_i + \epsilon_{ij} $$, $i = 1...a, j = 1, 2... n$

이 성립하고, 3.22에서는 treatment 종류인 $ a = 3 $, 각 treatment별 $n = 5$임을 알 수 있다.

\subsection{2-(a)}



이제, least squares normal equation을 세워보면,

$$S = \sum_{i=1}^{a} \sum_{j=1}^{n} (y_{ij} - \mu - \tau_i)^2 $$
위와 같을때, 위를 최소화하는 $\mu$, $\tau_i$들을 찾으면 된다.

즉, 위의 S를 각 $\mu$, $\tau_i$로 편미분할때 0으로 만드는 값들을 $\hat{\mu}$, $\hat{\tau_i}$들로 놓았을때 이들에 대한 식이 normal eqauation이 된다.

$\frac{\partial S}{\partial \mu} = 0$, $\frac{\partial S}{\partial \tau_i} = 0$에서,

$$N \hat{\mu} + n \hat{\tau_1} + ... + n \hat{\tau_a} = y_{..} $$
$$n \hat{\mu} + n \hat{\tau_i} = y_{i.}, \quad (i = 1, 2, ... a)$$

즉, 위의 a+1개의 식들이 normal equation이 되므로,

3.22의 데이터로 least square noraml equation을 세우면

$$15 \hat{\mu} + 5 \hat{\tau_1} + 5 \hat{\tau_2} + 5 \hat{\tau_3} = y_{..}$$
$$5 \hat{\mu} + 5 \hat{\tau_1} = y_{1.}$$
$$5 \hat{\mu} + 5 \hat{\tau_2} = y_{2.}$$
$$5 \hat{\mu} + 5 \hat{\tau_3} = y_{3.}$$
x
위와 같은 식들이 되고, $\sum_{i=1}^{3} \hat{\tau_i} = 0$인 constraint하에서, 이를 각 parameter에 대해 풀면

$$\hat{\mu} = \bar{y_{..}}$$
$$\hat{\tau_i} = \bar{y_{i.}} - \bar{y_{..}}, \quad (i = 1,2,3)$$
이 성립하게 된다.

따라서 이 데이터로 구한 $\tau_1 - \tau_2$의 추정값인 $\hat{\tau_1} - \hat{\tau_2}$의 값은 다음과 같다.

\begin{lstlisting}[language=R]
y1 <- c(9, 12, 10, 8, 15)
y2 <- c(20, 21, 23, 17, 30)
y3 <- c(6, 5, 8, 16, 7)
y <- c(y1,y2,y3)
mu <- mean(y)
tau_1 <- mean(y1) - mean(y)
tau_2 <- mean(y2) - mean(y)
tau_3 <- mean(y3) - mean(y)
answer <- tau_1 - tau_2
mu
tau_1
tau_2
tau_3
answer
\end{lstlisting}

위 코드의 실행결과에서

\begin{center}
    \includegraphics{hw2_2_a.jpg}
\end{center} 

-11.4가 답임을 알 수 있다. ($\hat{\mu} = 13.8, \hat{\tau_1} = -3, \hat{\tau_2} = 8.4, \hat{\tau_3} = -5.4$)



\subsection{2-(b)}

$$15 \hat{\mu} + 5 \hat{\tau_1} + 5 \hat{\tau_2} + 5 \hat{\tau_3} = y_{..}$$
$$5 \hat{\mu} + 5 \hat{\tau_1} = y_{1.}$$
$$5 \hat{\mu} + 5 \hat{\tau_2} = y_{2.}$$
$$5 \hat{\mu} + 5 \hat{\tau_3} = y_{3.}$$

constraint $\hat{\tau_3}=0$에서 위 normal equation을 풀면

$$\hat{\mu} =  \bar{y_{3.}}$$
$$\hat{\tau_1} = \bar{y_{1.}} - \bar{y_{3.}}$$
$$\hat{\tau_2} = \bar{y_{2.}} - \bar{y_{3.}}$$
$$\hat{\tau_3} = 0$$

위와 같아진다. 

이제, 각 parameter의 값을 구하면 다음과 같이 구할 수 있다.

\begin{lstlisting}[language=R]
y1 <- c(9, 12, 10, 8, 15)
y2 <- c(20, 21, 23, 17, 30)
y3 <- c(6, 5, 8, 16, 7)
y <- c(y1,y2,y3)
alt_mu <- mean(y3)
alt_tau_1 <- mean(y1) - mean(y3)
alt_tau_2 <- mean(y2) - mean(y3)
alt_tau_3 <- 0
answer <- tau_1 - tau_2
alt_mu
alt_tau_1
alt_tau_2
alt_tau_3
answer
\end{lstlisting}

위 코드의 실행결과에서

\begin{center}
    \includegraphics{hw2_2_b.jpg}
\end{center} 
로, $\tau_1 - \tau_2$의 추정값은 그대로 -11.4이고, 
$\hat{\mu} = 8.4, \hat{\tau_1} = 2.4, \hat{\tau_2} = 13.8, \hat{\tau_3} = 0$ 으로 달라졌음을 알 수 있다.

이 값이 달라진 이유로는, 아래와 같이 설명할 수 있다.

위 normal equation에서 constraint $\hat{\tau_3}=0$ 으로 두는것은, treatment 3의 평균을 기준으로 다른 treatment의 평균을 측정하겠다는 이야기와 같다.

즉, $y_{ij} = \mu_i + \epsilon_{ij}$일때, 2-(a)에서의 constraint가 의미하는것은 $\mu = \frac{\sum {i=1}^{3} \mu_i}{a}$를 기준으로 $\tau$들의 값을 결정하겠다는 이야기이고, 2-(b)에서의 constraint가 의미하는것은 $\mu = \mu_3$을 기준으로 
$\tau$들의  값을 결정하겠다는 이야기이다.

즉, constraint의 의미는 어떤 treatment를 기준으로 effect를 관측할것인가에 대한 의미인것이다.

따라서, treatment 1과 treatment 2의 차이인 $\tau_1 - \tau_2$ 이 constraint가 바뀌어도 그대로인 이유도 위와같이 설명이 가능하다.


\subsection{2-(c)}

$\mu + \tau_1$, $2\tau_1 - \tau_2 - \tau_3$, $\mu + \tau_1 + \tau_2$의 추정값은 각각 \\
$\hat{\mu} + \hat{\tau_1}$, $2\hat{\tau_1} - \hat{\tau_2} - \hat{\tau_3}$, $\hat{\mu} + \hat{\tau_1} + \hat{\tau_2}$로, 각 constraint에서 값은 다음과 같다.

\begin{center}
    \includegraphics{hw2_2_c.jpg}
\end{center} 

$$\sum_{i=1}^{3} \hat{\tau_i} = 0$$

$\hat{\mu} + \hat{\tau_1} = 10.8$
$2\hat{\tau_1} - \hat{\tau_2} - \hat{\tau_3} = -9$ 
$\hat{\mu} + \hat{\tau_1} + \hat{\tau_2} = 19.2$


$$\hat{\tau_3}=0$$

$\hat{\mu} + \hat{\tau_1} = 10.8$
$2\hat{\tau_1} - \hat{\tau_2} - \hat{\tau_3} = -9$ 
$\hat{\mu} + \hat{\tau_1} + \hat{\tau_2} = 24.6$

$\hat{\mu} + \hat{\tau_1}$ 의 의미는 treatment 1의 평균으로, 두 경우 모두 같다.
$2\hat{\tau_1} - \hat{\tau_2} - \hat{\tau_3}$ 의 의미는 treatment 1과 2, treatment 1과 3의 차이의 합으로, 두 경우 모두 같다.
 $\hat{\mu} + \hat{\tau_1} + \hat{\tau_2}$는 기준점에 treatment 1과 2의 효과를 더한것으로, 두 경우가 다르다.
기준점이 다르니 이는 당연한것이다.

\section{Q3} 

example 5의 모델은 다음과 같다.

$y_{ij} = \mu + \tau_i + \epsilon_{ij}$ (i = 1,2,3,4, j = 1,2,3..6)

이 모델을 이용해서, general regression significance test를 하자.

$H_0 : \tau_i = 0 \forall i = 1,2,3,4$, $H_1 : \tau_i \neq 0 \exists i$ 가 된다.

이를 test하기 위한 test statistic

$$F_0 = \frac{R(\tau|\mu) / (a - 1) }{ [\sum_{i=1}^{a} \sum_{j=1}^n y_{ij}^2 - R(\mu,\tau)] / N - a}$$

이고, 여기서 $a = 4, n = 6, N = 24$가 된다.

$H_0$하에서 $F_0 \sim F_{a-1, N-a}$가 성립하므로, 유의수준 $\alpha$에서 $F_0 > F_{a-1,N-a}(\alpha)$이면 $H_0$를 기각할것이다.

이제, test를 하기위해 $R(\tau|\mu)$, $R(\mu, \tau)$를 구해야한다.


$R(\tau|\mu) = R(\mu, \tau) - R(\mu)$이므로, 먼저 $R(\mu) $를 구하자.

general regression significance test를 위해, normal equation을 세우면,

$y_{ij} = \mu + \epsilon_{ij}$ 인 모델의 normal equation은

$$N \hat{\mu} = y_{..}$$

이므로, $R(\mu)$의 자유도는 normal equation중에 lin.indep한 식의 개수이므로 1이고,

$$ \hat{\mu} = \frac{y_{..}}{N} = \bar{y_{..}}$$에서

$$R(\mu) = \hat{\mu} * y_{..}$$ 이 된다.

이제 $R(\mu, \tau) $를 구하면

$y_{ij} = \mu + \tau_i + \epsilon_{ij}$인 모델의 normal equation은

$$ N \hat{\mu} + n \hat{\tau_1} + ... + n \hat{\tau_4} = y_{..}$$
$$N \hat{\mu} + n \hat{\tau_1} = y_{1.}$$
$$N \hat{\mu} + n \hat{\tau_2} = y_{2.}$$
$$N \hat{\mu} + n \hat{\tau_3} = y_{3.}$$
$$N \hat{\mu} + n \hat{\tau_1} = y_{4.}$$

이 되고, $R(\mu, \tau)$의 자유도는 normal equation중에 lin.indep한 식의 개수이므로 4가된다.

constraint $\sum_{i=1}^{4} \hat{\tau_i} = 0$에서,

$$ \hat{\mu} = \frac{y_{..}}{N} = \bar{y_{..}}$$
$$ \hat{\tau_i} = \bar{y_{i.}} - \bar{y_{..}} , (i = 1,2,3,4)$$ 이고,


$$R(\mu, \tau) = \hat{\mu} * y_{..} + \hat{\tau_1} * y_{1.} + \hat{\tau_2} * y_{2.} + \hat{\tau_3}* y_{3.} + \hat{\tau_4} * y_{4.} $$가 된다.


이제, 데이터를 이용하여 실제 $R(\mu, \tau) $, $R(\mu) $를 구하면 다음과 같다.

R 코드를 이용하여 구하면

\begin{lstlisting}[language=R]
y1 <- c(0.34, 0.12, 1.23, 0.70, 1.75, 0.12)
y2 <- c(0.91, 2.94, 2.14, 2.36, 2.86, 4.55)
y3 <- c(6.31, 8.37, 9.75, 6.09, 9.82, 7.24)
y4 <- c(17.15, 11.82, 10.95, 17.20, 14.35, 16.82)
y <- c(y1,y2,y3,y4)
y_dotdot <- sum(y)
y_1dot <- sum(y1)
y_2dot <- sum(y2)
y_3dot <- sum(y3)
y_4dot <- sum(y4)
mu_hat <- mean(y)
tau_1_hat <- mean(y1) - mean(y)
tau_2_hat <- mean(y2) - mean(y)
tau_3_hat <- mean(y3) - mean(y)
tau_4_hat <- mean(y4) - mean(y)

R_mu = mu_hat * y_dotdot
R_mu_tau = mu_hat * y_dotdot + tau_1_hat * y_1dot + tau_2_hat * y_2dot + tau_3_hat * y_3dot + tau_4_hat * y_4dot
R_tau_bar_mu = R_mu_tau - R_mu
sum_of_sqaure_y = sum(y**2)
F_0 =  (R_tau_bar_mu / 3) / ((sum_of_sqaure_y - R_mu_tau) / 20) 

(R_tau_bar_mu / 3)
((sum_of_sqaure_y - R_mu_tau) / 20) 
F_0
\end{lstlisting}

\begin{center}
    \includegraphics{hw2_3.jpg}
\end{center} 

각각 책의 값과 비교해보면,

$R(\tau | \mu) / 3$이 책의 $MS_{Trt}$인 236.1157,
$[\sum_{i=1}^{4} \sum_{j=1}^6 y_{ij}^2 - R(\mu,\tau)] / 20)$ 이 책의 $MS_{E}$와 같은 3.1041과 거의 같음을 알 수 있고,

$F_0$이 책에 있는 $F_0$ 인 76.07과 거의 같음을 알 수 있다.

유효숫자 표기에 의한 반올림을 고려하면, 사실상 책의 값과 똑같은 값이 나왔음을 알 수 있다.

귀무가설/대립가설이 같으므로 P-value도 같고, 따라서 example5와 똑같은 결론을 내릴 수 있다.

즉, usual한 ANOVA와 같은 결과가 나옴을 확인할 수 있었다.


\end{document}
